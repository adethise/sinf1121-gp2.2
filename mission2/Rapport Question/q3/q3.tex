\section*{Question 3 : Zacharie Kerger}
En consultant la documentation relative à la pile sur l'API de Java, on constate que la classe Stack étend la classe Vector avec 5 opérations supplémentaires afin de permettre à un vecteur d'adopter le comportement d'une pile.

\begin{itemize}

\item Des méthodes push et pop pour ajouter et retirer des éléments de la pile
\item Une méthode peek afin d'obtenir le premier élément de la pile sans le retirer
\item Une méthode empty qui vérifie si la pile est vide ou non
\item Une méthode search utilisée pour chercher un objet dans la pile et savoir à quelle distance il se trouve par rapport au premier élément

\end{itemize}

Cette classe ne peut pas convenir comme implémentation de l'interface Stack décrite dans DSAJ-5 parce qu'elle ne contient pas de méthode size et que la signature de la méthode push n'est pas la même que celle présente dans l'interface.


