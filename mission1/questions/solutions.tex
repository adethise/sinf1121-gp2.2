\documentclass[10pt,a4paper]{article}
\usepackage[utf8]{inputenc}
\usepackage[francais]{babel}
\usepackage[T1]{fontenc}
\usepackage{amsmath}
\usepackage{amsfonts}
\usepackage{amssymb}
\usepackage{graphicx}
\usepackage{enumitem}
\usepackage{lmodern}
\usepackage{listings}
\usepackage{color}

\definecolor{mygreen}{rgb}{0,0.6,0}
\definecolor{mygray}{rgb}{0.5,0.5,0.5}
\definecolor{mymauve}{rgb}{0.58,0,0.82}

\lstset{ %
  backgroundcolor=\color{white},   % choose the background color; you must add \usepackage{color} or \usepackage{xcolor}
  basicstyle=\footnotesize,        % the size of the fonts that are used for the code
  breakatwhitespace=false,         % sets if automatic breaks should only happen at whitespace
  breaklines=true,                 % sets automatic line breaking
  captionpos=b,                    % sets the caption-position to bottom
  commentstyle=\color{mygreen},    % comment style
  deletekeywords={...},            % if you want to delete keywords from the given language
  escapeinside={\%*}{*)},          % if you want to add LaTeX within your code
  extendedchars=true,              % lets you use non-ASCII characters; for 8-bits encodings only, does not work with UTF-8
  frame=single,                    % adds a frame around the code
  keepspaces=true,                 % keeps spaces in text, useful for keeping indentation of code (possibly needs columns=flexible)
  keywordstyle=\color{blue},       % keyword style
  language=Octave,                 % the language of the code
  morekeywords={*,...},            % if you want to add more keywords to the set
  numbers=left,                    % where to put the line-numbers; possible values are (none, left, right)
  numbersep=5pt,                   % how far the line-numbers are from the code
  numberstyle=\tiny\color{mygray}, % the style that is used for the line-numbers
  rulecolor=\color{black},         % if not set, the frame-color may be changed on line-breaks within not-black text (e.g. comments (green here))
  showspaces=false,                % show spaces everywhere adding particular underscores; it overrides 'showstringspaces'
  showstringspaces=false,          % underline spaces within strings only
  showtabs=false,                  % show tabs within strings adding particular underscores
  stepnumber=2,                    % the step between two line-numbers. If it's 1, each line will be numbered
  stringstyle=\color{mymauve},     % string literal style
  tabsize=2,                       % sets default tabsize to 2 spaces
  title=\lstname                   % show the filename of files included with \lstinputlisting; also try caption instead of title
}
\usepackage[left=2cm,right=2cm,top=2cm,bottom=2cm]{geometry}
\date{Vendredi 25 septembre 2014}
\author{Groupe 2.2}
\title{Premier rapport - Semaine 2}
\begin{document}
\maketitle
\section*{Question 1 : Thomas Celant}
Le type abstrait de données est un ensemble de données, contenant les méthodes et opérations permettant de les utiliser. La file ou la pile sont des types courants de TAD rencontrés.
Il est plus avantageux de décrire un TAD sur une interface, car il est par la suite plus facile de modifier l'implémentation de celui-ci. Cela rend donc aussi le TAD moins dépendant de son implémentation.

\section*{Question 2 : Charles Jacquet}
Premièrement, une liste simplement chaînée est une liste composée de noeuds, dans laquelle chaque noeud contient des données et la référence du noeud suivant (lié uniquement dans un sens). Le premier noeud est appelé la tête (head) et le dernier élément est appelé la queue (tail). La liste est référencée par un élément gardant la référence de la tête de la liste. La référence au noeud suivant (appelé next) de la queue est null.\\
Une pile est une liste dans laquelle on ne peut ajouter et supprimer des éléments qu'à la queue de la liste par des opérations appelées respectivement push et pop.\\
Implémentation des deux fonctions:
\begin{itemize}
\item {Push :}

Pour ajouter un élément, il faut parcourir la liste jusqu'à ce que l'élément de référence au prochain noeud soit null (la queue). Mettre la référence de ce noeud vers le nouveau noeud juste créé avec les données reçues et dont la référence au prochain noeud (next) est null.\\
Lorsque la liste est vide. Il suffit de créer le noeud (toujours avec le next = null) et de mettre la référence de début de liste vers ce noeud.\\
La complexité de cette fonction est de $\theta (n) $
\item {Pop : }

Pour supprimer un élément, il faut parcourir la liste jusqu'à ce que la référence contenue dans le noeud suivant soit null (next.next est null). Dans ce cas, il faut renvoyer les données de l'élément suivant et mettre la référence du noeud actuel (this.next) à null. 
Dans le cas où il n'y a qu'un seul élément (la tête = la queue) il faut simplement renvoyer les données de la tête et ensuite la supprimer.

La complexité de cette fonction est aussi de $\theta (n) $
\end{itemize}

On se rend compte que ce n'est pas très efficace car dans les deux cas, que ce soit pour supprimer ou ajouter un élément, il faut parcourir toute la liste ($\theta (n)$). 
Une liste doublement chaînée pourrait être beaucoup plus efficace si nous gardions une référence vers la queue. Dans ce cas, la complexité serait en $\theta (1)$.

\section*{Question 3}
\section*{Question 3 : Zacharie Kerger}
En consultant la documentation relative à la pile sur l'API de Java, on constate que la classe Stack étend la classe Vector avec 5 opérations supplémentaires afin de permettre à un vecteur d'adopter le comportement d'une pile.

\begin{itemize}

\item Des méthodes push et pop pour ajouter et retirer des éléments de la pile
\item Une méthode peek afin d'obtenir le premier élément de la pile sans le retirer
\item Une méthode empty qui vérifie si la pile est vide ou non
\item Une méthode search utilisée pour chercher un objet dans la pile et savoir à quelle distance il se trouve par rapport au premier élément

\end{itemize}

Cette classe ne peut pas convenir comme implémentation de l'interface Stack décrite dans DSAJ-5 parce qu'elle ne contient pas de méthode size et que la signature de la méthode push n'est pas la même que celle présente dans l'interface.



\section*{Question 4}
\section*{Question 5 : Lemaire Jerome}
Dans cette question, il était demandé de compléter l'interface se trouvant à la page 220 du DSAJ-6 en ajoutant les préconditions et les postconditions.\\
Voici l'interface complétée :\\


public interface Queue<E e> \left\{ \\
\\

\begin{size}


@pré : La file n'est pas modifiée pendant que la méthode est appliquée. \\ 
@post : Retourne le nombre d'éléments présents dans la file  si la file est vide alors la méthode retourne \textit{null} \\
 @exception : 

int size\left(\right);\\
\end{size}

\begin{isempty}

@pré : La file n'est pas modifiée pendant que la méthode est appliquée. \\
@post : Retourne \textit{true} si la file est vide sinon\textit{false} \\
 @exception : 

boolean isempty\left(\right);\\
\end{isempty}

\begin{enqueue}

@pré : \textbf{this} n'est pas modifiée pendant l'ajout de l'élément e.L'élément e est du même type que la liste.\\
@post :Ajout l'élément e à l'arrière de la file  \\
@exception : Si e n'est pas un élément du type attendu alors une InvalidElementException est lancée.

void enqueue  \left( E e\right)  throws InvalidElementException;\\
\end{enqueue}

\begin{first}

@pré : La file n'est pas modifiée durant l'application de la méthode. \\
@post : Retourne le premier élément de la file sans le retirer et retourne \textit{null} si la file est vide  \\
 @exception : 

E first\left(\right);\\
\end{first}

\begin{dequeue}

@pré : La file n'est pas modifiée durant l'application de la méthode. \\
@post : Retourne le premier élément de la file et le retire. Retourne \textit{null} si la file est vide  \\
 @exception : 

E dequeue\left(\right);\right\}\\



\end{dequeue}

\begin{explication}
Dans cette interface, les spécifications n'ont pas besoin d'être défensives car les méthodes n'autorisent pas l'utilisateur à modifier directement la pile sauf pour la méthode enqueue. La méthode enqueue doit donc vérifier si les préconditions sont respectées et lancer une exception en cas de non respect de celles-ci. Mais est-ce judicieux sachant qu'en java, une telle erreur lance automatiquement une exception ?
\end{explication}

\section*{Question 6 : Arnaud Dethise}

Soient deux files, $q_{1}$ et $q_{2}$.

push(A) : ajouter A dans la file non-vide ($q_{1}$).

pop() : transférer tous les éléments de la file non-vide ($q_{1}$) dans la file vide ($q_{2}$), à l'exception du dernier élément. Retourner le dernier élément. $q_{2}$ est maintenant la file non-vide, $q_{1}$ est la file vide.

\vspace{0.75cm}
Ainsi, les opérations suivantes :
\begin{lstlisting}
	push(1)
	push(2)
	push(3)
	a = pop()
	b = pop()
\end{lstlisting}
se traduisent en utilisant des files par :
\begin{lstlisting}
	q1.enqueue(1)                %   q1: [1]        q2: []
	q1.enqueue(2)                %   q1: [2,1]      q2: []
	q1.enqueue(3)                %   q1: [3,2,1]    q2: []
	q2.enqueue(p1.dequeue())     %   q1: [3,2]      q2: [1]
	q2.enqueue(p1.dequeue())     %   q1: [3]        q2: [2,1]
	a = q1.dequeue()             %   q1: []         q2: [2,1]
	q1.enqueue(p2.dequeue())     %   q1: [1]        q2: [2]
	b = q2.dequeue()             %   q1: [1]        q2: []
\end{lstlisting}

La complexité de la fonction push est alors de $\Theta(1)$ et celle de pop est de $\Theta(N)$ où N est le nombre d'élément dans la pile. Il est possible d'implémenter la pile au moyen de files de telle sorte que les complexités de push et pop soient inversées.

Cette solution est donc moins efficace que les implémentations normales d'une pile, dont la complexité est en temps constant pour push() et pop().

\vspace{0.75cm}
Nous avons également discuté en séance de la possibilité d'implémenter une file à partir de deux piles. La solution, pour laquelle la fonction push a une complexité constante $\Theta(1)$ et la fonction pop a une complexité moyenne $\Theta(1)$, est indiquée ci-dessous.

\begin{lstlisting}
	pile A
	pile B
	
	function enqueue(n) :
		A.push(n)
		
	function dequeue() :
		if B.isEmpty() :
			while not A.isEmpty() :
				B.push(A.pop())
		return B.pop()
\end{lstlisting}
\section*{Question 7 : Gil De Grove, Romain Henneton}
Fonction d'écriture:\\
\begin{lstlisting}

public static int WriteInFile(String filePath, String toWrite)
	{

		Path file = Paths.get(filePath);
		if (! file.toFile().exists())
		{
			try {		
				Files.createFile(file);
			} catch (IOException e) {
				e.printStackTrace();
			}

		}	
		if (! file.toFile().canWrite()){		
			return -1;
		}

		try {
			Files.write(file,toWrite.getBytes("US-ASCII"));
		} catch (UnsupportedEncodingException e) {
			e.printStackTrace();
		} catch (IOException e) {
			e.printStackTrace();
		}
		return 0;

	}



\end{lstlisting}

Fonction de lecture: \\
\begin{lstlisting}
public static List<String> ReadFromFile(String filePath)
	{
		List<String> text = new ArrayList<String>();
		Path file = Paths.get(filePath);
		if(! file.toFile().exists())
		{
			return text;
		}

		try {
			text = Files.readAllLines(file, Charset.forName("US_ASCII"));
		} catch (IOException e) {
			e.printStackTrace();
		}
		return text;
	}
	\end{lstlisting}
\end{document}
