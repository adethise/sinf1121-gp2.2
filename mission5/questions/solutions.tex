\documentclass[10pt,a4paper]{article}
\usepackage[utf8]{inputenc}
\usepackage[francais]{babel}
\usepackage[T1]{fontenc}
\usepackage{amsmath}
\usepackage{amsfonts}
\usepackage{amssymb}
\usepackage{graphicx}
\usepackage{enumitem}
\usepackage{lmodern}
\usepackage{listings}
\usepackage{color}

\definecolor{mygreen}{rgb}{0,0.6,0}
\definecolor{mygray}{rgb}{0.5,0.5,0.5}
\definecolor{mymauve}{rgb}{0.58,0,0.82}

\lstset{ %
  backgroundcolor=\color{white},   % choose the background color; you must add \usepackage{color} or \usepackage{xcolor}
  basicstyle=\footnotesize,        % the size of the fonts that are used for the code
  breakatwhitespace=false,         % sets if automatic breaks should only happen at whitespace
  breaklines=true,                 % sets automatic line breaking
  captionpos=b,                    % sets the caption-position to bottom
  commentstyle=\color{mygreen},    % comment style
  deletekeywords={...},            % if you want to delete keywords from the given language
  escapeinside={\%*}{*)},          % if you want to add LaTeX within your code
  extendedchars=true,              % lets you use non-ASCII characters; for 8-bits encodings only, does not work with UTF-8
  frame=single,                    % adds a frame around the code
  keepspaces=true,                 % keeps spaces in text, useful for keeping indentation of code (possibly needs columns=flexible)
  keywordstyle=\color{blue},       % keyword style
  language=Octave,                 % the language of the code
  morekeywords={*,...},            % if you want to add more keywords to the set
  numbers=left,                    % where to put the line-numbers; possible values are (none, left, right)
  numbersep=5pt,                   % how far the line-numbers are from the code
  numberstyle=\tiny\color{mygray}, % the style that is used for the line-numbers
  rulecolor=\color{black},         % if not set, the frame-color may be changed on line-breaks within not-black text (e.g. comments (green here))
  showspaces=false,                % show spaces everywhere adding particular underscores; it overrides 'showstringspaces'
  showstringspaces=false,          % underline spaces within strings only
  showtabs=false,                  % show tabs within strings adding particular underscores
  stepnumber=2,                    % the step between two line-numbers. If it's 1, each line will be numbered
  stringstyle=\color{mymauve},     % string literal style
  tabsize=2,                       % sets default tabsize to 2 spaces
  title=\lstname                   % show the filename of files included with \lstinputlisting; also try caption instead of title
}
\usepackage[left=2cm,right=2cm,top=2cm,bottom=2cm]{geometry}


\date{insert date 2014}
\author{Groupe 2.2}
\title{Produit Mission X}

\begin{document}
\maketitle

\section*{Question 1 Aurian de Potter}
\section*{Question 2 Romain Henneton}
L'algorithme de Knuth-Morris-Pratt (KMP) un un algorithme permettant de résoudre un problème appelé "pattern matching problem" sur un String, en général. Soit un texte P (pattern) de longeur m et un autre texte T de longueur n (avec m$\leq$n). L'algorithme KMP va permettre soit de déterminer si le pattern P figure dans le texte T et à quelle position, soit, le cas échéant, de déternimer que P ne figure pas dans T.



L'algorithme de brute force correspondant à cette recherche de pattern matching est constitué de deux boucles impriquées. La première (externe) parcourant les indices de T de 0 à n-m (via l'incrément i par exemple). La seconde (interne à la première), parcourant tous les éléments de P et comparant $T[i+j]$ ($0\leq j\leq n$) avec P$[j]$. La complexité de cette algorithme dans le pire des cas est donc O((n-m+1)n), donc O(nm).\\
L'algorithme de Knuth-Morris-Pratt est une variante améliorée de cette méthode de brute force. Lorsqu'un caractère ne correspondant pas au texte P est rencontré, plutot que de retourner à l'élément i+1 et de relancer la comparaison sur m, l'algorithme KMP va exploiter les comparaisons déjà réalisées et lancer la nouvelle recherche à un indice plus approprié et déterminé grâce à un pré-traitement de P. Ce pré-traitement est réalisé via une fonction appelée "Failure Function" qui associe à chaque indice de P une valeur définie comme le plus long préfixe P correspondant au sufixe de P[1..j]. Cela permet de redémarrer à un indice plus intéressant et d'éviter des comparaisons inutiles. La complexitée  est de O(n+m).
\section*{Question 3 Jérôme Lemaire}
\section*{Question 4 Zacharie Kerger}
\section*{Question 5 Arnaud Dethise}
\section*{Question 6 Arnaud Dethise}
%\section*{Question 7 Romain Henneton}
%\section*{Question 8 Zacharie Kerger}
%\section*{Question 9 Aurian de Potter}
%\section{Question 10 Jérome Lemaire}
% add sections if more questions

\end{document}
