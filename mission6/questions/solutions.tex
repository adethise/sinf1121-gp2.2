
\documentclass[10pt,a4paper]{article}
\usepackage[utf8]{inputenc}
\usepackage[francais]{babel}
\usepackage[T1]{fontenc}
\usepackage{amsmath}
\usepackage{amsfonts}
\usepackage{amssymb}
\usepackage{graphicx}
\usepackage{enumitem}
\usepackage{lmodern}
\usepackage{listings}
\usepackage{color}

\definecolor{mygreen}{rgb}{0,0.6,0}
\definecolor{mygray}{rgb}{0.5,0.5,0.5}
\definecolor{mymauve}{rgb}{0.58,0,0.82}

\lstset{ %
  backgroundcolor=\color{white},   % choose the background color; you must add \usepackage{color} or \usepackage{xcolor}
  basicstyle=\footnotesize,        % the size of the fonts that are used for the code
  breakatwhitespace=false,         % sets if automatic breaks should only happen at whitespace
  breaklines=true,                 % sets automatic line breaking
  captionpos=b,                    % sets the caption-position to bottom
  commentstyle=\color{mygreen},    % comment style
  deletekeywords={...},            % if you want to delete keywords from the given language
  escapeinside={\%*}{*)},          % if you want to add LaTeX within your code
  extendedchars=true,              % lets you use non-ASCII characters; for 8-bits encodings only, does not work with UTF-8
  frame=single,                    % adds a frame around the code
  keepspaces=true,                 % keeps spaces in text, useful for keeping indentation of code (possibly needs columns=flexible)
  keywordstyle=\color{blue},       % keyword style
  language=Octave,                 % the language of the code
  morekeywords={*,...},            % if you want to add more keywords to the set
  numbers=left,                    % where to put the line-numbers; possible values are (none, left, right)
  numbersep=5pt,                   % how far the line-numbers are from the code
  numberstyle=\tiny\color{mygray}, % the style that is used for the line-numbers
  rulecolor=\color{black},         % if not set, the frame-color may be changed on line-breaks within not-black text (e.g. comments (green here))
  showspaces=false,                % show spaces everywhere adding particular underscores; it overrides 'showstringspaces'
  showstringspaces=false,          % underline spaces within strings only
  showtabs=false,                  % show tabs within strings adding particular underscores
  stepnumber=2,                    % the step between two line-numbers. If it's 1, each line will be numbered
  stringstyle=\color{mymauve},     % string literal style
  tabsize=2,                       % sets default tabsize to 2 spaces
  title=\lstname                   % show the filename of files included with \lstinputlisting; also try caption instead of title
}
\usepackage[left=2cm,right=2cm,top=2cm,bottom=2cm]{geometry}


\date{5 Décembre 2014}
\author{Groupe 2.2}
\title{Réponse aux questions de la mission 6}

\begin{document}
\maketitle

\section*{Question 1 - Romain Henneton}
Un vertex est un noeud dans notre structure et un edge est une arête. Nous considérons n noeuds et m arêtes.
\begin{itemize}
\item revomeVertex retire un noeud, et les différentes étapes de suppression de ce noeud sont :
\begin{enumerate}
\item Supprimer le lien allant de tous les noeuds vers le noeud en question
\item Supprimer le noeud en question
\end{enumerate}
Nous voyons donc que la complexité dans le pire des cas est O(n) car il est nécessaire de parcourir tous les noeuds dans le cas où le noeud en question est connecté à tous les autres noeuds.
\item removeEdge supprime une arrête. Cependant, comme une arête relie nécessairement 2 noeuds, la complexité est O(1) vu qu'il suffit de supprimer les références dans ces deux noeuds.
\item insertVertex ajoute un noeud dans le graphe. Pour réaliser cette opération, il faut créer le noeud et le lier à un autre noeud. Nous avons donc encore une complexité O(1)
\end{itemize}

La réponse à cette question dépend de la structure de données utilisée pour mémoriser les objets contenant les sommets et les arêtes. Si ces objets contiennent des tableaux, alors chaque suppression ou ajout d'une arête entrainerait un redimensionnement d'un tableau, opération ayant comme complexité O(m) au lieu de O(1).

Le concept de location-aware entry est important pour justifier ces complexités. En effet, on sait retrouver la position d'une arête dans l'objet grâce au pointeur contenu dans l'objet vertex. L'ajout et la suppression d'un sommet dans le conteneur se fait donc en O(1) au lieu du O(n) nécessaire en cas de parcours de l'ensemble des éléments.
\section*{Question 2 - Zacharie Kerger}
Un parcours en largeur permet de parcourir un graphe de manière itérative. Cet algorithme va procéder de façon à ce que le parcours débute à un sommet initial et va ensuite visiter les sommets voisins, puis les autres sommets accessibles à partir de ces derniers, etc... Il est important de préciser qu'un sommet ne peut être visité qu'une seule fois.\\

Pour une matrice d'adjacence, l'algorithme va tout d'abord trouver le noeud initial dans la matrice. Ensuite, l'algorithme va devoir parcourir la ligne du noeud initial pour trouver les sommets voisins de celui-ci (complexité en $\Theta(n)$. Comme il y a $n$ sommets et que la matrice d'adjacence est une matrice carrée, l'algorithme va réaliser $n$ fois ce parcours de la ligne de chaque sommet. Par conséquent, la complexité temporelle de cet algorithme sera de $\Theta(n^{2})$

\section*{Question 3 - Lemaire Jérôme}
Les files peuvent être utilisées comme structure de données auxiliaire lors d'un tri topologique d'un graphe.\\


Premièrement, définissons un tri topologique, \\
Soit G un graphe orienté avec N sommets. Un ordre topologique de G est un ordre V1, V2, ... Vn des sommets de G tel que, pour chaque arête (Vi, Vj) de G, tel que i <j. Autrement dit, un ordre topologique est un ordre de telle sorte que ne importe quel trajet dirigé dans G traverse les sommets dans l'ordre croissant. Un graphe orienté peut avoir plus d'un ordre topologique.
Un tri topologique peut aussi être défini comme un ordre de visite des sommets tel qu'un sommet est toujours visité avec ses successeurs.\\

Etant donné qu'il existe différents tri topologique pour un même graphe acyclique, il se peut que le résultat du tri ne soit pas identique lors de l'utilisation d'une pile ou d'une file. Pour l'algorithme de tri topologique utilisant une file, le résultat peut différer pour un même arbre, par exemple si on ne démarre pas le tri par le même sommet. \\
Vous trouverez ci-dessous une brève explication d'une approche possible justifiant la faisabilité de celle-ci.\\
\subsection*{Explication de l'algorithme de tri topologique d'une file}
Le concept est le suivant, si on travaille sur un arbre acyclique, il existe toujours un sommet du graphe ne possédant pas de vecteur entrant. Une fois que l'on a repéré ce sommet, on le supprime du graphe et on l'ajoute dans la file de tri. Nous avons un nouveau (sous)-graphe, toujours acyclique et donc contenant un nouveau sommet sans vecteur entrant, on le supprime du graphe et on l'ajoute dans la file de tri. On répète cette opération jusqu'à ce que le sous-graphe soit vide. La file de tri contient bien les sommets trier de manière topologique.


\section*{Question 4 - Aurian de Potter}
L'algorithme proposé ne fonctionnera que dans quelque cas, en effet à chaque fois que l'algorithme choisit un chemin avec le poids le plus court il va se déplacer vers le noeud suivant et perdre le noeud précédent ce qui va créer des situations on il sera bloqué. Comme illustré ci-dessous.

\begin{center}
    \includegraphics[scale=0.7]{ex.png}
\end{center}

\section*{Question 5 - Arnaud Dethise}

	L'objectif de cette question est d'écrire un algorithme permettant de trouver le meilleur chemin dans un graphe. 
	Il est donc logique d'utiliser l'algorithme de Dijkstra, modifié pour tenir compte des méthodes spécifiques à ce problème dans le calcul du meilleur chemin.
	
	Les deux différences suivantes sont à prendre en compte :
	\begin{itemize}
	\item Le meilleur chemin est celui ayant \textit{la plus grande} bande passante.
	\item Le coût d'un chemin est égal au \textit{coût de l'arête la plus faible} sur ce chemin.
	\end{itemize}
	
	\begin{lstlisting}
	MaxBandWidth(G, a, b) :
		Initialize D[a] = %*$\infty$*) and D[v] = 0 for each vertex v %*$\neq$*) a
		Let a priority queue Q contain all the vertices of G using the D labels as keys
		while b is in Q do
			u = Q.removeMax()
			for each edge (u,v) such that v is in Q do
				if min(D[u], bw(u,v)) > D[v] then
					D[v] = min(D[u], bw(u,v))
					Change the key of vertex v in Q to D[v]
		return D[b]
	\end{lstlisting}

\section*{Question 6 - Arnaud Dethise}

	Remarque : la complexité justifiée ci-dessous est la même que celle de l'algorithme de Dijkstra.

	Soit G contenant $n$ nœuds et $m$ arêtes.
	
	\vspace{0.3cm}
	Première solution :
	\begin{itemize}
	\item Initialiser les valeurs de la liste D : $\mathcal{O}(n)$
	\item Initialiser la priorityQueue Q (heap) : $\mathcal{O}(n~log~n)$
	\item Récupérer $n$ fois l'élément maximal de Q : $\mathcal{O}(n)$
	\item Trouvez tous les vecteurs reliant u à un nœud encore dans Q (matrice d'adjacence) : $\mathcal{O}(m)$
	\item Mettre à jour $m$ fois les clés des vecteurs de Q : $\mathcal{O}(m~log~n)$
	\end{itemize}
	Total : $\mathcal{O}((n+m)~log~n)$, avec $m \leq n^2$.
	
	\vspace{0.3cm}	
	Deuxième solution :
	\begin{itemize}
	\item Initialiser les valeurs de la liste D : $\mathcal{O}(n)$
	\item Initialiser la priorityQueue Q (unsorted list, location aware) : $\mathcal{O}(n)$
	\item Récupérer $n$ fois l'élément maximal de Q : $\mathcal{O}(n^2)$
	\item Trouvez tous les vecteurs reliant u à un nœud encore dans Q (matrice d'adjacence) : $\mathcal{O}(m)$
	\item Mettre à jour $m$ fois les clés des vecteurs de Q : $\mathcal{O}(m)$
	\end{itemize}
	Total : $\mathcal{O}(n^2 + m)$, avec $m \leq n^2$, soit $\mathcal{O}(n^2)$.
	
	\vspace{0.5cm}
	Remarque : Le DSAJ-6 indique qu'une implémentation utilisant la structure \textit{Fibonacci heap} permettrait d'obtenir une complexité asymptotique en $\mathcal{O}(m + n~log~n)$, mais une rapide recherche sur le web indique que cette structure, bien que asymptotiquement plus efficace, est en pratique plus lourde par clé à cause de ses facteurs constants.

% add sections if more questions

\end{document}
